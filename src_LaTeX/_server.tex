%=======================================================================
% File         : _server.tex
% Title        : 
% Author(s)    : Sebastien VARRETTE <Sebastien.Varrette@uni.lu>
% Creation     : 06 Jul 2009
% Time-stamp:    <2009-07-15 14:23:18 svarrette>
% $Id: _server.tex 447 2009-07-15 13:36:11Z svarrette $
% 
%   Copyright (c) 2009 Sebastien Varrette. 
%
% Licence: Creative Commons Attribution-Noncommercial-Share Alike 2.0 France
%
% You are free:
%   - to Share:  to copy, distribute and transmit the work
%   - to Remix: to adapt the work
%
% Under the following conditions:
%   - Attribution : You must attribute the work in the manner specified by the
%                   author or licensor (but  not in any way that suggests that
%                   they endorse you or your use of the work).  
%   - Noncommercial:  You may not use this work for commercial purposes.
%   - Share Alike: If you alter, transform, or build upon this work, you may
%                  distribute the resulting work only under the same or similar
%                  license to this one. 
%
% With the understanding that:
%   - Waiver: Any of the above conditions can be waived if you get permission from
%             the copyright holder. 
%   - Other Rights: In no way are any of the following rights affected by the
%                   license:  
%                    * Your fair dealing or fair use rights;
%                    * Apart from the remix rights granted under this license,
%                      the author's moral rights; 
%                     * Rights other persons may have either in the work itself
%                       or in how the work is used, such as publicity or privacy
%                       rights. 
%   - Notice:  For any reuse or distribution, you must make clear to others the
%              license terms of this work. The best way to do this is with a
%              link to this web page. 
%=======================================================================

%------------------------------
\subsection{Basic installation}
\label{sec:server:install}

This section describes the installation of a server under a Linux Debian
machine. 
First proceed to the Linux installation as stated in \S\ref{sec:urt_client}. 
Then create a dedicated user (\texttt{urbanterror}) for running the server,
attached to the group \texttt{urt}: 
\begin{lstlisting}[style=command]
[root@host]# adduser --system --ingroup urt urbanterror
\end{lstlisting}
The next step is to make this user owning the files coming with Urban Terror,
and creating the directory storing the logs and the pid files: 
\begin{lstlisting}[style=command]
[root@host]# chown -R urbanterror:urt /usr/local/games/urbanterror
[root@host]# mkdir /var/log/urbanterror /var/run/urbanterror
[root@host]# chown urbanterror:urt /var/log/urbanterror /var/run/urbanterror
[root@host]# ln -s ~urbanterror/.q3a/q3ut4/urt.log  /var/log/urbanterror/urt.log
\end{lstlisting}

As the server will be launched under the rights of the \texttt{urbanterror}
user, the log files of the server will be located  in
\verb!~urbanterror/.q3a/q3ut4/urt.log!. 
Le last instruction creates a symbolic link for the future log file of the
server in a more convenient place. 

\noindent You now have to decide about which server executable you want to use (either 32
bits \texttt{ioUrTded.i386} or 64 bits \texttt{ioUrTded.x86\_64}), make it
executable and create the link to it. 
It follows that for running a 32 bits executable, you will typically run the
following commands: 
\begin{lstlisting}[style=command]
[root@host]# cd /usr/local/games/urbanterror
[root@host]# chmod +x  ioUrTded.i386
[root@host]# ln -s ioUrTded.i386 urbanterror.server  
\end{lstlisting}

\noindent If you followed exactly this tutorial, the configuration file of the server
resides in  
\texttt{/usr/local/games/urbanterror/q3ut4/server.cfg}. 
It is nicely commented so you shouldn't have trouble to adapt it to reflect your
own configuration. 
Yet here are the variables you should pay attention: 
\begin{lstlisting}[style=urtcfg]
sets " Admin" "yourname" 
sets " Email" "xxx@xxx.xxx"

set sv_hostname "your server name, by xxx" 
set g_motd "Your stats on xxx" //Your message of the day here, 
set sv_joinmessage "Welcome to Urban Terror 4.1"  //Your joinmessage here

sets sv_dlURL "urt.unfoog.de" // use this server instead of teh default one

set g_gametype "8" // default to bomb mode ;)

set rconpassword "xxx" //Password to control the server remotely using rcon.

set sv_master2 ""  // leave those empty to prevent notifications of master 
set sv_master3 ""  // servers about your server. 
set sv_master4 ""

set g_log "urt.log" // name of the log file
//*** B3 Specific settings ***
set g_logSync "3"     //XLR: Unbuffered/direct logging for B3
set sv_zombietime "6" //XLR: Larger zombietime to reduce slot/client mixups for B3
set g_loghits "1" //XLR: headshotcounter and XLRstats bodypart information for B3
\end{lstlisting}

The \texttt{server.cfg} I'm currently using for my own server is proposed in
appendix~\ref{anx:server.cfg}, page~\pageref{anx:server.cfg}\\

Now you should be able to start the server. On Debian, you should be addicted to
have a script for startup in the directory \texttt{/etc/init.d/}.
It appeared difficult to find one so I made one proposed in
appendix~\ref{anx:init.d:urbanterror} page~\pageref{anx:init.d:urbanterror}. 
Put this script in the folder \texttt{/etc/init.d/}, make it
executable (\texttt{chmod +x /etc/init.d/urbanterror}). You can now start the
server by issuing the command: 
\begin{lstlisting}[style=command]
[root@host]# /etc/init.d/urbanterror start
\end{lstlisting}

Now your server is running and should be seen from the client software of your
network. Check that you can play on the server, vote to cycle, change map etc. 
Once ensured that the server is working fine, you can continue to
section~\ref{sec:b3} to configure Big Brother Bot and make your server even more  
powerful. 

%--------------------------------------------------------------------------
\subsection{Installing new maps and adapting the \texttt{mapcycle.txt} file}
\label{sec:server:new:map}

To install a new map on your server, you have several steps to do: 
\begin{enumerate}
\item copy the map file (typically \texttt{ut4\_mapname.pk3}) into the folder\\
  \texttt{/usr/local/games/urbanterror/q3ut4/}; 
\item ensure the map file is available for auto-download on the server you
  configured with the directive \texttt{sv\_dlURL} in \texttt{server.cfg} (see
  \S\ref{sec:server:install});
\item adapt your \texttt{mapcycle.txt} file to include your new map in the map
  list (see my script \texttt{createmapcycles.sh} below);
\item change the \texttt{map} directive at the end of your \texttt{server.cfg}
  to  \texttt{ut4\_mapname} (remove the extension) such that n restart, the new
  map will be proposed directly;
\item restart the server by issuing the following command: 
  \begin{lstlisting}[style=command]
    [root@host]# /etc/init.d/urbanterror restart
  \end{lstlisting}
\end{enumerate}

\noindent To automatically handle new maps and randomizing the maps in the
\texttt{mapcycle.txt} file, I wrote two scripts: 
\begin{itemize}
\item \texttt{createmapcycles.sh}, to be put together with the file
  \texttt{mapcycle.txt.orig} in the directory
  \texttt{/usr/local/urbanterror/q3ut4/} (both proposed in
  appendix~\ref{anx:mapcycle:all} page~\pageref{anx:mapcycle:all}). \\
  This script is responsible for creating a file named \texttt{mapcycle.txt.all}
  listing the default maps (taken from \texttt{mapcycle.txt.orig}) and all newly
  added maps (detecting by a simple \texttt{ls} command on the files in the
  directory with format \texttt{ut4\_*}).
\item \texttt{randomize\_mapcycle.pl} that randomizes the entries of a file
  listing the maps to be put in the final \texttt{mapcycle.txt}. 
  This Perl script should be placed in the directory
  \texttt{/usr/local/urbanterror/q3ut4/}.  
\end{itemize}
 
The first script (\texttt{createmapcycles.sh}) is typically launched once for
each newly installed map, \emph{i.e.} between the steps 4 and 5 described at the
beginning of this section. 

On the contrary, the second script is typically launched every time the server
is started. If you want to activate this feature, simply uncomment the line
defining the variable \texttt{MAPCYCLE\_RANDOMIZE\_SCRIPT} in the startup script
proposed in appendix~\ref{anx:init.d:urbanterror}.


%~~~~~~~~~~~~~~~~~~~~~~~~~~~~~~~~~~~~~~~~~~~~~~~~~~~~~~~~~~~~~~~~
% eof
%
% Local Variables:
% fill-column: 80
% mode: latex
% mode: flyspell
% mode: auto-fill
% End:
