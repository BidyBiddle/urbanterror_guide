%=======================================================================
% File         : _b3.tex
% Title        : 
% Author(s)    : Sebastien VARRETTE <Sebastien.Varrette@uni.lu>
% Creation     : 06 Jul 2009
% Time-stamp:    <Thu 2012-07-26 11:35 svarrette>
%   
%   Copyright (c) 2009 Sebastien Varrette. 
%
% Licence: Creative Commons Attribution-Noncommercial-Share Alike 2.0 France
%
% You are free:
%   - to Share:  to copy, distribute and transmit the work
%   - to Remix: to adapt the work
%
% Under the following conditions:
%   - Attribution : You must attribute the work in the manner specified by the
%                   author or licensor (but  not in any way that suggests that
%                   they endorse you or your use of the work).  
%   - Noncommercial:  You may not use this work for commercial purposes.
%   - Share Alike: If you alter, transform, or build upon this work, you may
%                  distribute the resulting work only under the same or similar
%                  license to this one. 
%
% With the understanding that:
%   - Waiver: Any of the above conditions can be waived if you get permission from
%             the copyright holder. 
%   - Other Rights: In no way are any of the following rights affected by the
%                   license:  
%                    * Your fair dealing or fair use rights;
%                    * Apart from the remix rights granted under this license,
%                      the author's moral rights; 
%                     * Rights other persons may have either in the work itself
%                       or in how the work is used, such as publicity or privacy
%                       rights. 
%   - Notice:  For any reuse or distribution, you must make clear to others the
%              license terms of this work. The best way to do this is with a
%              link to this web page. 
%=======================================================================

Big Brother Bot B3~\cite{b3} is a complete and total server administration
package for online games. 
B3 is based on the analysis of the log files populated by a running server. 
It means that as soon as B3 is able to parse and interpret the logs of a given
game, this game is supported. This is the case for Urban Terror. 
The main interest of B3 resides in the fact that the administrations tasks are
handled by plugins written by the community yet with a common interface and
installation procedure. This makes B3 very flexible and easy to use. 

I detail here the configuration of B3 on your Linux server, assuming you
followed the instructions proposed in the previous section.

Note that your main source of information should be the official guidelines
provided at the following url:\\ 
\url{http://wiki.github.com/BigBrotherBot/big-brother-bot/installation}

If you experience any problem, don't hesitate to use B3 forum available at 
\url{http://www.bigbrotherbot.com/forums} 

%--------------------------
\subsection{Prerequisites}
\label{sec:b3:prerequisite}

Several components are required before installing B3 in itself: 
\begin{itemize}
\item A MySQL server and therefore PhPMyAdmin and a Web server (the later being
  mandatory for \texttt{xlrstats}).
  Under Debian, just run: 
  \begin{lstlisting}[style=command]
[root@host]# apt-get install apache2 mysql-server phpmyadmin
[root@host]# ln -s /usr/share/phpmyadmin/ /var/www/phpmyadmin
  \end{lstlisting}
  For the configuration, see below. 
\item some Python stuff (\url{http://www.python.org}), more particularly: 
  Python 2.3+, Elementtree, Python-MySQL and Python Setuptools. Under Debian,
  simply run: 
  \begin{lstlisting}[style=command]
[root@host]# apt-get install python2.5 python-elementtree \ 
             python-mysqldb python-setuptools
  \end{lstlisting}
\end{itemize}

%.....................................
\subsubsection{Web server configuration}
\label{sub:b3:apache}

As you plan to use phpMyAdmin, it is mandatory to activate SSL support for the
Web server, which means you will have to create a certificate for it.
This requires \texttt{OpenSSL} so install it: 
  \begin{lstlisting}[style=command]
[root@host]# apt-get install openssl
  \end{lstlisting}

Then get the fully qualified host name of the server (run \texttt{hostname -f}
typically), or use the future url of your server. 
In the sequel, I'll assume the server to be reached by the name
\texttt{myurtserver.domain.com}. 
To create the certificates (valid for 1414 days), run:  
  \begin{lstlisting}[style=command]
[root@host]# cd /etc/apache2/
[root@host]# mkdir ssl.key ssl.crt ssl.crl
[root@host]# chmod 700 ssl.key
[root@host]# openssl req -new -x509 -days 1414 -nodes \ 
   -keyout /etc/apache2/ssl.key/server.key -out /etc/apache2/ssl.crt/server.crt \ 
   -subj '/O=COM/OU=DOMAIN/CN=myurtserver.domain.com'
  \end{lstlisting}
I don't know how familiar you are with SSL certification, yet you should adapt
the subject of the certificate (precised with \texttt{-subj}) to reflect your
own structure (for the \texttt{O}rganization name (\texttt{COM} here) and your
\texttt{O}rganization \texttt{U}nit name (\texttt{DOMAIN} here). The critical
point is to place your server fully qualified name as \texttt{C}ommon
\texttt{N}ame.
For more information, read the \texttt{README} file coming with apache2: 
\begin{lstlisting}[style=command]
  [root@host]# zless /usr/share/doc/apache2.2-common/README.Debian.gz
\end{lstlisting}

This creates a self-signed certificate \texttt{/etc/apache2/ssl.crt/server.crt}
and the associated private key \texttt{/etc/apache2/ssl.crt/server.key}. 
Now you should notify the SSL module of Apache about the location of those
certificates. 
Edit \texttt{/etc/apache2/mods-available/ssl.conf} and add/update the
properties as follows: 
\begin{lstlisting}[style=apachecfg]
<IfModule mod_ssl.c>
  ...
  # The certificates of the server to use
  SSLCertificateFile      /etc/apache2/ssl.crt/server.crt
  SSLCertificateKeyFile   /etc/apache2/ssl.key/server.key
  ...
</IfModule>
\end{lstlisting}
Now comment the \texttt{NameVirtualHost} directive in \texttt{/etc/apache2/ports.conf}.
Then activate the SSL module (together with the rewrite one): 
\begin{lstlisting}[style=command]
[root@host]#  a2enmod ssl
[root@host]#  a2enmod rewrite
\end{lstlisting}

\noindent You should now update the file configuring the web server by editing
the file \texttt{/etc/apache2/site-available/default} as follows:  
\lstinputlisting[caption=The \texttt{/etc/apache2/site-available/default}
configuration file for your Apache server,style=apachecfg,basicstyle=\scriptsize]{../config_files/apache2/site-default}

\noindent Now start the web server by running:
\begin{lstlisting}[style=command]
[root@host]# apache2ctl restart
\end{lstlisting}
(You can also use "\texttt{/etc/init.d/apache2 restart}"). Check that the server
works by opening a browser in the url \url{http://myurtserver.domain.com}. 
You should be automatically redirect to the url
\url{https://myurtserver.domain.com}.  

%.....................................
\subsubsection{MySQL configuration}
\label{sub:b3:mysql}

Assuming the MySQL server to be running (\texttt{/etc/init.d/mysql restart}
eventually), you should be able also to use PHPMyAdmin by accessing to the url 
\url{https://myurtserver.domain.com/phpmyadmin/}.
Note that the server is configured the authorize the access to this interface
only for the machine with the IP address \texttt{your.ip.adress.here} according
to configuration of the Apache server (see \S\ref{sub:b3:apache}). 
In case of trouble, your very first reflex (as always) is to check the problem
in the logs. To check them online, run in a separate terminal (CTRL-C to quit): 
\begin{lstlisting}[style=command]
[root@host]# tail -f /var/log/apache2/error.log
\end{lstlisting}

In all case, once you're able to interact with the MySQL server, your very first
task is to setup a root password for it (blank by default). 
Prefer using PHPMyAdmin for that (go to the Privilege section) or run the
following command (just ensure you flush by after the history of the commands to
avoid password retrieval -- this is not perfect but a minimum security measure):
\begin{lstlisting}[style=command]
[root@host]# mysqladmin -u root password "your_password_here"
[root@host]# history -c
\end{lstlisting}

Using PHPMyAdmin, create a MySQL user \texttt{b3} (generate a random password
for him and keep it in a safe place such as an encrypted file) with full right
on a database named \texttt{b3}. 
Create also a MySQL user \texttt{b3\_ro} with read-only access to the database
\texttt{b3} (\emph{i.e.} with privileges limited to the \texttt{SELECT}
command). Again, generate a random password for him and keep it in a safe
place. 
In the sequel, I'll assume those users to be configured as follows: 
\begin{itemize}\setitemsep
\item MySQL user \texttt{b3}: password \texttt{\_\_b3\_password\_\_};
\item MySQL user \texttt{b3\_ro}: password \texttt{\_\_b3\_ro\_password\_\_}.
\end{itemize}
 
%--------------------------
\subsection{B3 installation}
\label{sec:b3:install}

You are now ready to install B3 (When writing this tutorial, the latest release
of B3 was the version 1.1.4b). 
Simply run:
\begin{lstlisting}[style=command]
[root@host]# easy_install -U b3
\end{lstlisting}
\emph{Note:} this will install b3 in the following directory: \\
\texttt{/usr/lib/python2.5/site-packages/b3-1.1.4b-py2.5.egg/b3} \\
In the sequel and to make things simpler (and shorter), I will refer to this
directory as \begginstall.

%...................................
\subsubsection{Preparing the database}
\label{sec:b3:dbsetup}

Retrieve the file \begginstall\texttt{/doc/b3.sql} from the server (typically
using \texttt{scp}). 
Then, use PHPMyAdmin to connect to the database \texttt{b3} (created during the process
described in \S\ref{sub:b3:mysql}) as user \texttt{b3}.
Select the menu \texttt{Import}, then choose the previously retrieved file,
\texttt{b3.sql} and finally click on \texttt{Go}. 
This should populate the database \texttt{b3} with the tables required by Big
Brother Bot.  

%...........................
\subsubsection{B3 Configuration}
\label{sec:b3:config}

Create the directory structure for hosting the B3 configuration as follows: 
\begin{lstlisting}[style=command]
[root@host]# mkdir ~urbanterror/b3
[root@host]# cd ~urbanterror/b3
[root@host]# mkdir -p conf extplugins/conf init.d plugins_toinstall
[root@host]# cp $B3_EGG_INSTALLDIR/conf/* ~urbanterror/b3/conf/
[root@host]# ln -s /var/log/urbanterror/b3.log b3.log
[root@host]# chown -R urbanterror:urt ~urbanterror/b3
\end{lstlisting}%$

The main configuration file for B3 is named \texttt{b3.xml} and is located in
the directory \texttt{/home/urbanterror/b3/conf/}. 
Edit and adapt it as the one proposed in appendix~\ref{anx:b3:b3.xml}
page~\pageref{anx:b3:b3.xml}. 
In particular, it is very important to disable PunkBuster otherwise you will get
the error message "\emph{[pm] Please try your command after you have been
  authenticated}" when playing on the server and trying to register.

The configuration proposed in appendix~\ref{anx:b3:b3.xml} activate a single
plugin called \texttt{admin}. 
This is for a first check and we will see later how to add and activate new
plugins. 

The only step missing for the installation is a startup script to manage B3 
\texttt{/etc/init.d/bigbrotherbot}.  
Mine is proposed in appendix~\ref{anx:init.d:b3} page~\pageref{anx:init.d:b3}.  
Assuming your UrT server is running (\emph{i.e.} the log files are populated),
you should be able to launch b3 as follows: 
\begin{lstlisting}[style=command]
[root@host]# /etc/init.d/bigbrotherbot start
\end{lstlisting}

To check this is working, connect to the server and once in the game, type as a
message \verb/!register/. This will register yourself in B3 (assuming your
player name is \texttt{Toto}, you should receive as an answer in the console the
fact that \texttt{Toto} has been put in the \texttt{User} group. 
In all case, detect the anomalies by monitoring the log files of b3 and running
in a separate terminal:
\begin{lstlisting}[style=command]
[root@host]# tail -f /var/log/urbanterror/b3.log
\end{lstlisting}

\noindent 
You shall now familiarize yourself with the commands available under B3 with the
plugin \texttt{admin}. 
For this, you shall refer to the official manual available at: 
\texttt{http://wiki.github.com/BigBrotherBot/big-brother-bot/manual-commands}\\

You will notice that each command comes with a level to authorize or not the
access to the command. 
To make yourself admin of B3 (\emph{i.e.} able to launch all commands), the best
is to update the table \texttt{clients} in the \texttt{b3} database. 
Locate the entry with your player name (field \texttt{name}). 
The value associated to your name in the \texttt{group\_bits} fields should be
1. This translate the fact that you have been put in the User group. 
Change this value to 128 to make you an admin of B3. 
%
You might want to delegate some administration tasks to other players. 
For this, make their \texttt{group\_bits} to 20 typically. 

%--------------------------------------------
\subsection{Configuring the default plugins}
\label{sec:b3:plugins}

B3 comes with a few plugins more or less relevant for Urban Terror summarized in
Table~\ref{tab:b3:plugins}. 
\begin{table}[ht]
  \centering\small
  \begin{tabular}{|c|p{0.8\textwidth}|}
    \hline
    \texttt{admin} & Provides the majority of B3 functionality such as kicking,
    banning, user management, and warnings\\
    \texttt{adv} & Advertise messages periodically\\
    \texttt{censor} & Warn users when using banned words\\
    \texttt{stats} & Simple transient stats that track stats while the users
    is connected\\ 
    \texttt{pingwatch} & Warn users for high/low pings and kick when the ping
    goes to high\\ 
    \texttt{spamcontrol} & Warn users for saying too many messages in a short
    period of time\\ 
    \texttt{status} &  Dumps an XML file to the server periodically with B3 and
    user information to be used on your website to display game status \\
    \texttt{tk} & Monitors team damage and team kills\\ 
    \texttt{welcome} &Welcome messages for new and returning users\\\hline 
  \end{tabular}
  \caption{Default plugins proposed with Big Brother Bot}
  \label{tab:b3:plugins}
\end{table}

\noindent Until now, B3 is configured with the \texttt{admin} plugin enabled. 
Before going further, a few elements should be known about the plugins for B3:
\begin{itemize}
\item A given plugin \texttt{toto} is written in Python in a file named
  \texttt{toto.py}. This file is located in \begginstall\texttt{/plugins/} (this
  holds for the default plugins).
  On startup, B3 eventually compile this plugin to generate a new file
  \texttt{toto.pyc} for a more efficient execution. 
  Therefore if for any reason you decide to modify the source file of a plugin,
  remember to delete the compiled file before restarting B3. Otherwise your
  modification won't be taken into account.
  You may claim you will nether modify plugins sources files, yet it may happens
  if you want to change the position in the screen where a given message is
  displayed. You can alter this location as follows:
  \begin{itemize}
  \item Upper left (server announce area): 
    \texttt{self.console.write('blabla')} or \texttt{self.console.broadcast('blabla')}
  \item Lower left (chat area): \texttt{self.console.say('blabla')}
  \item Center screen (Big text): \texttt{self.console.write('bigtext blabla')}
  \end{itemize}



\item A given  plugin \texttt{toto} comes with a configuration file
  \texttt{plugins\_toto.xml} that respects the following format: 
  \begin{lstlisting}[language=xml,numbers=none,]
<configuration plugin="toto">
  ...
</configuration>    
  \end{lstlisting}
  This configuration files is located by default in \begginstall\texttt{/conf/}
  yet remember that I preferred to put the default configurations files in    
  \texttt{/home/urbanterror/b3/conf/}.
  Actually, the exact location of plugins configuration files is specified by
  the \texttt{config} attribute of the \texttt{plugin} anchor used in the
  configuration file \texttt{b3.xml}.
\item External plugins are handled in a separate directory, either in the
  default location \begginstall\texttt{/extplugins/} or, in the hierarchy
  proposed in this document, in \texttt{/home/urbanterror/b3/extplugins/}.\\
  This directory will contain the Python sources files of the plugins together
  with a sub-directory \texttt{conf/} which host the plugins configuration
  files.
\item Activating a plugin is done through the configuation file \texttt{b3.xml} in
  the \texttt{plugins} section.  
  Each plugin is characterized by a name, a unique priority value (within the
  list of your plugins) and a path to
  the configuration file as follows: 
  \begin{lstlisting}[style=apachecfg]
<plugins>
   <plugin name="plugin_name" priority="n"  config="/path/to/plugin_conf.xml"/>
</plugins>
  \end{lstlisting}
  The name provided should correspond to the basename of the Python source file
  (without the extension \texttt{.py}).  
  B3 will search for such a file in the default installation directory
  (\emph{i.e.} \begginstall\texttt{/}), then in the default external plugin
  directory (\emph{i.e.}  \begginstall\texttt{/extplugins/}) and finally in a
  user-defined external directory specified in a \texttt{settings} section of
  \texttt{b3.xml}. \\
  For instance, in the configuration files proposed in
  appendix~\ref{anx:b3:b3.xml}, we setup this external directory to
  \texttt{/home/urbanterror/b3/extplugins} as follows:
  \begin{lstlisting}[style=apachecfg]
<settings name="plugins">
   <set name="external_dir">/home/urbanterror/b3/extplugins</set>
</settings>
  \end{lstlisting}
  Note that when specifying the path to the plugin configuration file, you can
  use the sequence \texttt{@b3/} to refer to \begginstall\texttt{/}. 
\end{itemize}

\noindent You can now activate one of the default plugins summarized in the
table~\ref{tab:b3:plugins}.  For instance to activate the plugin
\texttt{welcome}: 
\begin{enumerate}\setitemsep
\item Examine \texttt{/home/urbanterror/b3/conf/plugin\_welcome.xml} to
  eventually adapt the configuration to suits your needs;
\item Select a priority value $n$ (3 in the sequel) and add a new
  \texttt{plugin} anchor in the \texttt{plugins} section of \texttt{b3.xml}:
  \begin{lstlisting}[style=apachecfg]
<plugins>
   ...
   <plugin name="welcome" priority="3" 
           config="/home/urbanterror/b3/conf/plugin_welcome.xml"/>
</plugins>
  \end{lstlisting}
\item Restart B3:
\begin{lstlisting}[style=command]
[root@host]# /etc/init.d/bigbrotherbot restart
\end{lstlisting}
\end{enumerate}

%--------------------------------------------------------------------------
\subsection{General procedure to install third party plugins}
\label{sec:b3:extplugins}

The flexibility of B3 permits the availability of various third party plugins. 
You will find them in the \texttt{Plugins/Releases} section of the Big Brother Bot forum
(indeed a very nice source of info for you): 
\begin{center}
  \url{http://www.bigbrotherbot.com/forums/index.php?board=17.0}
\end{center}

\noindent 
On this page, you should find a list of released plugins, each of them coming
with a short description, a download location and an install guide. 
You shall always refer to the install guide yet here is the general procedure to
install and configure a third plugin. 
To make this procedure concrete, I'll illustrate it on the \texttt{heashotsurt}
plugin (version \texttt{0.2.0}) which counts the number of headshot kills made
by the players: 
\begin{enumerate}
\item Download the plugin sources \texttt{heashotsurt-v0.2.0.zip} and unzip this
  file into the folder:\\
  \texttt{/home/urbanterror/b3/plugins\_toinstall/heashotsurt-v0.2.0}
\item Copy (or make a symbolic link of) the Python source file
  \texttt{heashotsurt.py} into the external plugin directory 
  \texttt{/home/urbanterror/b3/extplugins/}
\item Copy (or make a symbolic link of) the  configuration file
  \texttt{headshots.xml} into the  external plugin configuration directory\\ 
  \texttt{/home/urbanterror/b3/extplugins/conf/}
\item As for default plugins, select a unique priority value $n$ (14 for
  instance) and add a new \texttt{plugin} anchor in the \texttt{plugins} section
  of \texttt{b3.xml}:
  \begin{lstlisting}[style=apachecfg]
   <plugin name="headshotsurt" priority="14"  
           config="/home/urbanterror/b3/extplugins/conf/headshots.xml"/>
  \end{lstlisting}
\item Restart B3:
\begin{lstlisting}[style=command]
[root@host]# /etc/init.d/bigbrotherbot restart
\end{lstlisting}
\end{enumerate}

\noindent This procedure holds for most of the external plugins. 
I personally installed in this way the following plugins: 
\begin{itemize}\setitemsep
\item \texttt{PowerAdminUrt} (version \texttt{1.4.0b3})\\
  {\small \url{http://www.bigbrotherbot.com/forums/index.php?topic=426.0}}\\
  Remember to edit the file \texttt{extplugins/conf/poweradminurt.xml} and
  disable the \texttt{rotationmanager} by putting the value False as follows:
  \begin{lstlisting}[style=apachecfg]
<settings name="rotationmanager">
    <!-- enable the rotationmanager? -->
    <set name="rm_enable">False</set>
    ...
</settings> 
 \end{lstlisting}
\item \texttt{headshotsurt} (version \texttt{0.2.0}) \\ 
  {\small \url{http://www.bigbrotherbot.com/forums/index.php?topic=367.0}}
\item \texttt{topkiller} (version \texttt{1.0.4})\\
  {\small\url{http://www.bigbrotherbot.com/forums/index.php?topic=574.0}}
\item \texttt{Spree} (version \texttt{1.1})\\
  {\small\url{http://www.bigbrotherbot.com/forums/index.php?topic=864.0}}
\end{itemize}

I now dedicated a full section to one of the most interesting external plugin
which comes with B3: \texttt{xlrstats}

%--------------------------------------------------------------------------
\subsection{Installing XLRstats}
\label{sec:b3:xlrstats}

XLRstats (\url{http://www.xlrstats.com/}) is a statistics plugin for
BigBrotherBot (B3) and it stores all kill-events in the MySQL database used by
B3.
Stats are available in real time in the game using the \texttt{!xlrstats} command in
chat, but much more can be viewed in the XLRstats webfront (such as weapon
usage, ranks, medals etc.).
The sources of the plugin are part of B3 installation (we won't touch them). 
You still need to download the following elements (put them
into \verb!~urbanterror/b3/plugins_toinstall/!):
\begin{itemize}\setitemsep
\item The latest webfront \texttt{xlrstats-web-v2} (version 2.0.6), available on\\
 \url{http://github.com/xlr8or/xlrstats-web-v2/tree};
\item The image pack for UrT (\texttt{xlrstats-imagepack-urt}), to get from\\
  \url{http://xlr8or.snt.utwente.nl/forum/index.php?action=downloads}. 
\end{itemize}

\noindent 
You first need to create the extra tables in the \texttt{b3} database. 
The procedure is similar to the one explained in \S\ref{sec:b3:dbsetup}, this
time using the file present in the sub-directory 
\texttt{xlrstats-web-v2/xlrstats/sql/xlrstats.sql}

\noindent 
Once this is done, two B3 plugins should be installed: 
\begin{enumerate}
\item \texttt{status}, one of the default plugins (see \S\ref{sec:b3:plugins}). \\
  Just edit \texttt{/home/urbanterror/b3/conf/plugin\_status.xml} as follows:
  \begin{lstlisting}[style=apachecfg]    
<configuration plugin="status">
	<settings name="settings">
		<set name="interval">60</set>
		<set name="output_file">/var/www/status.xml</set>
	</settings>
</configuration>
  \end{lstlisting}
  This will configure the plugin to populate the file
  \texttt{/var/www/status.xml} for which the web server should have read
  access. In this purpose, run the following commands: 
  \begin{lstlisting}[style=command]
[root@host]# touch /var/www/status.xml
[root@host]# chown urbanterror:www-data /var/www/status.xml
[root@host]# chmod 644 /var/www/status.xml
  \end{lstlisting}
\item the \texttt{ctime} plugin that comes with XLRStats webfront in
  the zip file \texttt{b3-plugins/ctime.zip}. Once unpacked, 
  create the extra tables in the \texttt{b3} database using the
  \texttt{ctime.sql} file (re-use the procedure proposed in
  \S\ref{sec:b3:dbsetup}). 
  Then configure the plugin as explained in \S\ref{sec:b3:extplugins}.
\end{enumerate}

\noindent As always, the plugins are enabled in B3 as soon as you add them in the
\texttt{plugins} section of \texttt{b3.xml} (just adapt the priority values eventually): 
  \begin{lstlisting}[style=apachecfg]
<plugins>
   ...
   <plugin name="xlrstats" priority="2"  config="@b3/extplugins/conf/xlrstats.xml"/>
   ...
   <plugin name="status" priority="10"  
           config="/home/urbanterror/b3/conf/plugin_status.xml"/>
   <plugin name="ctime"  priority="11"
           config="/home/urbanterror/b3/extplugins/conf/plugin_ctime.xml"/>
   ...
</plugins>
  \end{lstlisting}

\noindent 
Restart B3 and ensure everything is working correctly (check the log file \\
\texttt{/var/log/urbanterror/b3.log} eventually). 
Now we can setup the webfront (take a look at \url{http://www.xlrstats.com/} for
the reference webfront documentation). 
First copy the files of the webfront into a more convenient place:
\begin{lstlisting}[style=command]
[root@host]# cd /home/urbanterror/b3/plugins_toinstall/xlrstats-web-v2/xlrstats
[root@host]# cp -r xlrstats /usr/share/
[root@host]# cd ../../
[root@host]# cp -r xlrstats-imagepack-urt/images/* /usr/share/xlrstats/images/
[root@host]# chown -R urbanterror:www-data /usr/share/
[root@host]# chmod -R g+r /usr/share/xlrstats
[root@host]# ln -s /usr/share/xlrstats/ /var/www/xlrstats
\end{lstlisting}

\noindent 
Then make the folders \texttt{dynamic} and \texttt{config} writable by the
webserver:
\begin{lstlisting}[style=command]
[root@host]# chmod 770 /var/www/xlrstats/dynamic
[root@host]# chmod 770 /var/www/xlrstats/config
\end{lstlisting}

\noindent 
Update \texttt{/etc/apache2/sites-available/default} to include
the \texttt{xlrstats} directory (typically to limit access to it to a few IP addresses):
\begin{lstlisting}[style=apachecfg]
<Directory /var/www/xlrstats/>
        Order Deny,Allow
        Deny from all
	Allow from 127.0.0.1
        Allow from toto.ip.address
	Allow from tata.ip.address	
</Directory>

<VirtualHost xx.yy.zz.tt:443>
        ...
        Alias /stats  /var/www/xlrstats
        ...
</VirtualHost>

\end{lstlisting}
Finally restart the web server: 
\begin{lstlisting}[style=command]
[root@host]# apache2ctl restart
\end{lstlisting}

\noindent 
Now go to the url \url{http://myurtserver.domain.com/stats/install/}
Select "\emph{Start Install}", then "\emph{Next}". 
In the "MySQL Connection Settings" panel, enter the following configuration (adapt the password): 
\begin{itemize}\setitemsep
\item Host name: \texttt{localhost}
\item MySQL Username : \texttt{b3\_ro}
\item MySQL Password: \texttt{\_\_b3\_ro\_password\_\_}
\item MySQL Database Name: \texttt{b3}
\end{itemize}
In the "Basic Game \& Server Settings", make sure you select the game Urban
Terror, enter the public IP of the server (\texttt{xx.yy.zz.tt:27960} -- see
\S\ref{sub:b3:apache}).  The B3 status url is
\url{http://myurtserver.domain.com/status.xml}

Once everything is done, a file \texttt{config/statsconfig.php} is created. 
Ensure this file has the permissions \texttt{750}. 
You should now go to the url\\
\url{http://myurtserver.domain.com/stats/config/install_award_idents.php} to
setup the medals etc. (actually, you should do it every time you experience a
trouble with awards). You can now delete the \texttt{install} directory and
enjoy XLRStats at \url{http://myurtserver.domain.com/stats/}. 


%~~~~~~~~~~~~~~~~~~~~~~~~~~~~~~~~~~~~~~~~~~~~~~~~~~~~~~~~~~~~~~~~
% eof
%
% Local Variables:
% fill-column: 80
% mode: latex
% mode: flyspell
% mode: auto-fill
% End:
